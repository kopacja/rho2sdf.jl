%% LyX 2.3.7 created this file.  For more info, see http://www.lyx.org/.
%% Do not edit unless you really know what you are doing.
\documentclass[english]{article}
\usepackage[T1]{fontenc}
\usepackage[latin9]{inputenc}
\usepackage{amsmath}
\usepackage{babel}
\begin{document}
Consider finite element spatial discretization of the position and
density

\begin{equation}
\boldsymbol{x}(\boldsymbol{\xi})=\sum_{a=1}^{n_{\mathrm{en}}}N_{a}(\boldsymbol{\xi})\boldsymbol{x}_{a}
\end{equation}

\begin{equation}
\rho(\boldsymbol{\xi})=\sum_{a=1}^{n_{\mathrm{en}}}N_{a}(\boldsymbol{\xi})\rho_{a}
\end{equation}
Further, let us introduce a squared distance function as

\begin{equation}
d(\boldsymbol{\xi})=\frac{1}{2}\left\Vert \boldsymbol{x}(\boldsymbol{\xi})-\boldsymbol{x}_{g}\right\Vert ^{2}
\end{equation}
The problem at hand can be formulated as

\begin{equation}
\begin{cases}
\text{find} & \boldsymbol{\xi}=\arg\min d(\boldsymbol{\xi})\\
\text{subjected to} & \rho(\boldsymbol{\xi})=\rho_{\mathrm{t}}\\
 & \xi_{i}=\bar{\xi}
\end{cases}
\end{equation}
This constrained problem can be refolmulated as an unconstrained one
using the method of Lagrange multipliers. To this end, let us introduce
the Lagrangian as

\begin{equation}
\mathcal{L}(\boldsymbol{\xi},\lambda)=d(\boldsymbol{\xi})+\lambda_{1}\left(\rho(\boldsymbol{\xi})-\rho_{\mathrm{t}}\right)+\lambda_{2}\left(\xi_{i}-\bar{\xi}\right)
\end{equation}
Its extremal point $(\boldsymbol{\xi},\boldsymbol{\lambda})$ have
to satisfy

\begin{align}
\frac{\partial\mathcal{L}(\boldsymbol{\xi},\lambda)}{\partial\boldsymbol{\xi}} & =\frac{\partial d(\boldsymbol{\xi})}{\partial\boldsymbol{\xi}}+\frac{\partial\rho(\boldsymbol{\xi})}{\partial\boldsymbol{\xi}}\lambda_{1}+\frac{\partial\xi_{i}}{\partial\boldsymbol{\xi}}\lambda_{2}=0\\
\frac{\partial\mathcal{L}(\boldsymbol{\xi},\lambda)}{\partial\lambda_{1}} & =\rho(\boldsymbol{\xi})-\rho_{\mathrm{t}}=0\\
\frac{\partial\mathcal{L}(\boldsymbol{\xi},\lambda)}{\partial\lambda_{2}} & =\xi_{i}-\bar{\xi}=0
\end{align}
This is a system of non-linear algebraic equation which can be resolved
for instance by the Newton-Raphson method. To this end one must perform
the Taylor expansion around point $(\boldsymbol{\xi}_{k},\lambda_{k})$
up to first order

\begin{align}
\frac{\partial\mathcal{L}(\boldsymbol{\xi}_{k},\boldsymbol{\lambda}_{k})}{\partial\boldsymbol{\xi}}+\frac{\partial^{2}\mathcal{L}(\boldsymbol{\xi}_{k},\boldsymbol{\lambda}_{k})}{\partial\boldsymbol{\xi}^{2}}\Delta\boldsymbol{\xi}+\frac{\partial^{2}\mathcal{L}(\boldsymbol{\xi}_{k},\boldsymbol{\lambda}_{k})}{\partial\boldsymbol{\xi}\partial\lambda_{1}}\Delta\lambda_{1}+\frac{\partial^{2}\mathcal{L}(\boldsymbol{\xi}_{k},\boldsymbol{\lambda}_{k})}{\partial\boldsymbol{\xi}\partial\lambda_{2}}\Delta\lambda_{2} & =0\\
\frac{\partial\mathcal{L}(\boldsymbol{\xi}_{k},\boldsymbol{\lambda}_{k})}{\partial\lambda_{1}}+\frac{\partial^{2}\mathcal{L}(\boldsymbol{\xi}_{k},\boldsymbol{\lambda}_{k})}{\partial\lambda_{1}\partial\boldsymbol{\xi}}\Delta\boldsymbol{\xi}+\frac{\partial^{2}\mathcal{L}(\boldsymbol{\xi}_{k},\boldsymbol{\lambda}_{k})}{\partial\lambda_{1}^{2}}\Delta\lambda_{1}+\frac{\partial^{2}\mathcal{L}(\boldsymbol{\xi}_{k},\boldsymbol{\lambda}_{k})}{\partial\lambda_{1}\partial\lambda_{2}}\Delta\lambda_{2} & =0\\
\frac{\partial\mathcal{L}(\boldsymbol{\xi}_{k},\boldsymbol{\lambda}_{k})}{\partial\lambda_{2}}+\frac{\partial^{2}\mathcal{L}(\boldsymbol{\xi}_{k},\boldsymbol{\lambda}_{k})}{\partial\lambda_{2}\partial\boldsymbol{\xi}}\Delta\boldsymbol{\xi}+\frac{\partial^{2}\mathcal{L}(\boldsymbol{\xi}_{k},\boldsymbol{\lambda}_{k})}{\partial\lambda_{2}\partial\lambda_{1}}\Delta\lambda_{1}+\frac{\partial^{2}\mathcal{L}(\boldsymbol{\xi}_{k},\boldsymbol{\lambda}_{k})}{\partial\lambda_{2}^{2}}\Delta\lambda_{2} & =0
\end{align}
which can be written in matrix notation as

\begin{equation}
\left[\begin{array}{ccc}
\frac{\partial^{2}\mathcal{L}_{k}}{\partial\boldsymbol{\xi}^{2}} & \frac{\partial^{2}\mathcal{L}_{k}}{\partial\boldsymbol{\xi}\partial\lambda_{1}} & \frac{\partial^{2}\mathcal{L}_{k}}{\partial\boldsymbol{\xi}\partial\lambda_{2}}\\
\frac{\partial^{2}\mathcal{L}_{k}}{\partial\lambda_{1}\partial\boldsymbol{\xi}} & \boldsymbol{0} & \boldsymbol{0}\\
\frac{\partial^{2}\mathcal{L}_{k}}{\partial\lambda_{2}\partial\boldsymbol{\xi}} & \boldsymbol{0} & \boldsymbol{0}
\end{array}\right]\left\{ \begin{array}{c}
\Delta\boldsymbol{\xi}\\
\Delta\lambda_{1}\\
\Delta\lambda_{2}
\end{array}\right\} =-\left\{ \begin{array}{c}
\frac{\partial\mathcal{L}_{k}}{\partial\boldsymbol{\xi}}\\
\frac{\partial\mathcal{L}_{k}}{\partial\lambda_{1}}\\
\frac{\partial\mathcal{L}_{k}}{\partial\lambda_{2}}
\end{array}\right\} 
\end{equation}

In the sequal, all required derivatives will be derived

\textbf{
\begin{equation}
\frac{\partial^{2}\mathcal{L}(\boldsymbol{\xi}_{k},\boldsymbol{\lambda}_{k})}{\partial\lambda_{1}^{2}}=\frac{\partial^{2}\mathcal{L}(\boldsymbol{\xi}_{k},\boldsymbol{\lambda}_{k})}{\partial\lambda_{2}^{2}}=\frac{\partial^{2}\mathcal{L}(\boldsymbol{\xi}_{k},\boldsymbol{\lambda}_{k})}{\partial\lambda_{1}\partial\lambda_{2}}=0
\end{equation}
}

\begin{equation}
\frac{\partial^{2}\mathcal{L}(\boldsymbol{\xi},\boldsymbol{\lambda})}{\partial\lambda_{1}\partial\boldsymbol{\xi}}=\frac{\partial\rho(\boldsymbol{\xi})}{\partial\boldsymbol{\xi}}
\end{equation}

\begin{equation}
\frac{\partial^{2}\mathcal{L}(\boldsymbol{\xi},\boldsymbol{\lambda})}{\partial\lambda_{2}\partial\boldsymbol{\xi}}=\frac{\partial\xi_{i}}{\partial\boldsymbol{\xi}}
\end{equation}

\begin{align}
\frac{\partial^{2}\mathcal{L}_{k}}{\partial\boldsymbol{\xi}^{2}} & =\frac{\partial}{\partial\boldsymbol{\xi}}\left(\frac{\partial d(\boldsymbol{\xi})}{\partial\boldsymbol{\xi}}+\frac{\partial\rho(\boldsymbol{\xi})}{\partial\boldsymbol{\xi}}\lambda_{1}+\frac{\partial\rho_{i}}{\partial\boldsymbol{\xi}}\lambda_{2}\right)\\
 & =\frac{\partial^{2}d(\boldsymbol{\xi})}{\partial\boldsymbol{\xi}^{2}}+\frac{\partial^{2}\rho(\boldsymbol{\xi})}{\partial\boldsymbol{\xi}^{2}}\lambda_{1}
\end{align}

where

\begin{align}
\frac{\partial d(\boldsymbol{\xi})}{\partial\boldsymbol{\xi}} & =\frac{\partial}{\partial\boldsymbol{\xi}}\left(\frac{1}{2}\left\Vert \boldsymbol{x}(\boldsymbol{\xi})-\boldsymbol{x}_{g}\right\Vert ^{2}\right)\\
 & =\frac{\partial}{\partial\boldsymbol{\xi}}\left(\frac{1}{2}\left\{ \boldsymbol{x}(\boldsymbol{\xi})-\boldsymbol{x}_{g}\right\} \cdot\left\{ \boldsymbol{x}(\boldsymbol{\xi})-\boldsymbol{x}_{g}\right\} \right)\\
 & =\frac{\partial\boldsymbol{x}(\boldsymbol{\xi})}{\partial\boldsymbol{\xi}}\cdot\left\{ \boldsymbol{x}(\boldsymbol{\xi})-\boldsymbol{x}_{g}\right\} 
\end{align}

and

\begin{align}
\frac{\partial^{2}d(\boldsymbol{\xi})}{\partial\boldsymbol{\xi}^{2}} & =\frac{\partial}{\partial\boldsymbol{\xi}}\left(\frac{\partial\boldsymbol{x}(\boldsymbol{\xi})}{\partial\boldsymbol{\xi}}\cdot\left\{ \boldsymbol{x}(\boldsymbol{\xi})-\boldsymbol{x}_{g}\right\} \right)\\
 & =\frac{\partial^{2}\boldsymbol{x}(\boldsymbol{\xi})}{\partial\boldsymbol{\xi}^{2}}\cdot\left\{ \boldsymbol{x}(\boldsymbol{\xi})-\boldsymbol{x}_{g}\right\} +\frac{\partial\boldsymbol{x}(\boldsymbol{\xi})}{\partial\boldsymbol{\xi}}\cdot\frac{\partial\boldsymbol{x}(\boldsymbol{\xi})}{\partial\boldsymbol{\xi}}
\end{align}

\end{document}
